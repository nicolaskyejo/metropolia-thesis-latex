\documentclass[12pt, a4paper]{article}

\usepackage{amsmath}
\usepackage{fontspec}
\usepackage{geometry}
\usepackage{setspace}
\usepackage{luacode}
\usepackage{color}

\setmainfont{Times New Roman}
\geometry{margin=1.5cm}
\linespread{1.46}
\definecolor{metroorange}{RGB}{185, 98, 30}

\title{{\color{metroorange}Executive Summary} \\

Implementation of a Forensic Analysis System for Malicious Network Traffic}
\author{{\color{metroorange}Nicolas Kyejo}}
\date{\directlua{metropoliadate()}}
\setlength{\parindent}{0pt} % Remove paragraph indentation
\setlength{\parskip}{16.5pt} %one empty line to separate paragraph

\begin{luacode}
function metropoliadate()
    local today = os.date("!\%d \%B \%Y")
    tex.sprint(today)
end
\end{luacode}

\begin{document}
    \maketitle
    \thispagestyle{empty}

    \newpage
    \setcounter{page}{1}


    \section{Background}\label{sec:background}

    The main goal of the final year project and thesis was to examine and create a forensic system for detecting malicious network traffic.
    The justification for such a project was to see if it could be used reliably and also be a low-cost security solution.
    Some examples of network forensic systems include Intrusion Prevention System (IPS), Intrusion Detection System (IDS) and Security
    Information and Event Management (SIEM).
    These tools are generally expensive; a cursory search on the internet shows that the average price of an intrusion detection system is over 5000 EUR for hardware-based solutions and over 1000 EUR for software-based solutions.
    Therefore, a supporting goal was to see if a similar system could be built with freely available tools and open data.

    The implemented forensic system closely resembles an IDS in functionality since it only predicts (detects) malicious network traffic; it does not prevent malicious traffic from reaching the target network.
    The system was implemented with the help machine learning methods to classify network traffic as malicious or benign.


    \section{Methodology}\label{sec:methodology}

    Initially, the forensic system solution was meant to not use any machine learning methods.
    The initial planned solution was to use programming constructs to decide if a piece of network traffic was malicious or not---this would be similar to what `Expert Systems' do in practice.
    However, after some research, it was found that machine learning methods could be used to create a more robust solution in less time.

    The first step was to find a suitable dataset to train the machine learning models;
    through manual searching, the \textit{CICIDS2017} dataset was found among various existing IDS datasets.
    The main reason for choosing this dataset was that it was relatively small compared to other datasets encountered.
    Additionally, it was also well documented and had software to extract features from raw packet capture data (PCAP).

    The next step was to find suitable supervised machine learning methods commonly used in classification tasks.
    Among the numerous machine learning methods existing (variants also), four were chosen for the implementation: \textit{K-Nearest Neighbors} (KNN), \textit{Random Forest}, \textit{Logistic Regression}, and \textit{Multi-Layer Perceptron} (MLP).

    The training and evaluation of the machine learning was done using the \textit{scikit-learn} library in Python.
    The process consisted of loading the dataset, splitting it into training and testing sets, training the models (for each algorithm), and noting the test score accuracy.
    Furthermore, the hyperparameters of the algorithms were tuned manually, and the best parameters were noted.
    For the hyperparameters chosen, the models were trained again with cross-validation to ensure that the test score results were not overfitting on the training data.
    The models trained were then saved to disk using the \textit{skops} library in order to be used later for the final evaluation.


    Data for the final evaluation was captured from a virtualized environment created using \textit{Vagrant} and \textit{VirtualBox}.
    The environment consisted of three Kali Linux guest machines: \textit{defender}, \textit{attacker}, and \textit{target}.
    All three machines were in the same network to make implementation easier in regard to capturing network traffic.
    The capture was done with \textit{tcpdump} and the resulting \textit{PCAP} files were converted (feature extraction) to \textit{CSV} files with the help of \textit{cicflowmeter}.
    The benign captured network traffic consisted of HTTP web browsing, FTP, SSH, telnet, and DNS query traffic.
    On the other hand, the malicious traffic consisted of denial-of-service (ICMP ping flood), brute force login, SQL injection, command injection, and XSS attacks.

    The final implementation step was to develop a user interface for predicting whether a piece of network traffic was malicious or not.
    This step involved giving a PCAP file as input and outputting a prediction of either `Benign' or `Malicious'.
    To facilitate this step, a GUI was created using \textit{tkinter}, a Python wrapper to the Tk GUI toolkit.
    For multiple predictions, a simple script was created to automate the process of predicting multiple PCAP files.


    \section{Findings}\label{sec:findings}

    As a precursor to the final prediction on the captured network traffic, the performance of the models was first compared with each other.
    The comparisons were done using learning curves, confusion matrices, mean F1-scores, and time taken to train the models.
    The Random Forest model performed the best even though it took the longest to train;
   the KNN model took the shortest time to train while performing the worst in terms of accuracy.
    Overall, the results showed that the models performed well in terms of accuracy and training time.

    The final prediction on the captured network traffic was done using all four models.
    The prediction process consisted of predicting each captured traffic whose label was known and the output recorded for each model's prediction.
    The final results showed that the KNN model performed the best in terms of accuracy while the Random Forest model performed the worst;
    the runner-ups were the MLP and Logistic Regression models.
    The only model with a false-positive prediction was the MLP model, which predicted web browsing benign traffic as malicious.
    The Random Forest model was not able to detect any of the attacks in the captured traffic.
    This could be due to the model overfitting to the training data.

    \clearpage

    One of the attacks---the command injection attack---was detected by all models except the Random Forest model despite the attack not being in the \textit{CICIDS2017} dataset.
    This result showed that the models were able to generalize well to a new attack, which is one of the advantages of using machine learning methods.


    \section{Conclusion}\label{sec:conclusion}

    In the context of the thesis project aims, it was able to achieve them using open-source tools and open data;
    the model training was done on a mid-range laptop hardware released in 2020.
    The trained models using supervised machine learning methods proved effective in detecting malicious traffic.

    However, some limitations with the project were noted, such as the limited number of attack type scenarios examined and the lack of real-world network testing for the trained models.
    One recommendation for further research is the development of open datasets for creating network forensic solutions through crowdsourcing network traffic data and logs;
    Privacy concerns for such data could be addressed with anonymization measures.

\end{document}

