\documentclass[11pt,a4paper,oneside]{article}
\title{Implementation of a Forensic Analysis System for Malicious Network Traffic\\Thesis Plan}
\author{Nicolas Kyejo}

%packages%

\usepackage{luacode}
\usepackage[numbers,square]{natbib}
\usepackage[english]{babel}
\usepackage{amsmath}
\usepackage{amssymb}
\usepackage{xcolor}
\usepackage{hyperref}        %for links
\hypersetup{colorlinks,breaklinks,
    urlcolor=[RGB]{0,0,0},
    linkcolor=[RGB]{0,0,0},
    citecolor=[RGB]{0,0,0}
}
\usepackage{url}
\urlstyle{same}
\usepackage{graphicx}
\usepackage{fontspec}
\usepackage{float}
\setmainfont{Arial}


\linespread{1.5} %line spacing

\makeatletter
\renewcommand{\@dotsep}{10000}
\makeatother


%margins
\usepackage[top=2.5cm, bottom=3cm, left=4cm, right=2cm, nofoot]{geometry}
\setlength{\parindent}{0pt} %first line of paragraph not indented
\setlength{\parskip}{16.5pt} %one empty line to separate paragraph

\usepackage{fancyhdr}     %package for fancy header and footers
\usepackage{lastpage}     %records the last page number
\pagestyle{fancy}
\fancyhf{} % sets both header and footer to nothing
\renewcommand{\headrulewidth}{0pt}
\usepackage{wallpaper}

% your new footer definitions here  
\cfoot{}
\lfoot{}
\rhead{\thepage\ (\pageref{LastPage})}

\makeatletter
\renewcommand\@biblabel[1]{#1\hspace{1cm}} %from [1] to 1 with 1cm gap
\makeatother

\begin{luacode}
function metropoliadate()
    local today = os.date("!\%d \%B \%Y")
    tex.sprint(today)    
end
\end{luacode}

\begin{document}

    \makeatletter
    \renewcommand{\maketitle}{
        \newgeometry{left=4.5cm}
        \thispagestyle{empty}
        \ThisCenterWallPaper{1}{old_cover}
        \vspace*{9.5cm}
        {\Large\@author\\[1cm]\LARGE{\color[RGB]{0,0,0}\@title}}
        \null\vfill
        \parbox{.7\linewidth}{
            Metropolia University of Applied Sciences\\
            Bachelor of Engineering\\
            Information Technology\\
            \directlua{metropoliadate()} % i.e 20 January 2020
        }

        \ThisLRCornerWallPaper{1}{old_metropolia}
        \restoregeometry
        \clearpage
    }
    \makeatother

    \maketitle
    \newpage

    % Uncomment the following lines if you want to include a table of contents
    \thispagestyle{empty}
    \ThisLRCornerWallPaper{1}{old_footer}
    \tableofcontents
    \clearpage
    \newpage
    \setcounter{page}{1}

    \section{Introduction}\label{sec:introduction}

    The aim of this document is outline a plan for conducting the work of the bachelor thesis \textit{``Implementation of a Forensic Analysis System for Malicious Network Traffic''}.
    The specialization of the degree program is \textit{``IoT and Cloud computing''}.
    The goal of the thesis is to explore existing solutions for performing forensic analysis on captured network traffic data,
    and also implement a similar system to evaluate its effectiveness and whether it could be useful as a low cost solution.

    \section{Approach}\label{sec:approach}
    The planned approach to achieve the thesis project's goal is to do the following:
    \begin{itemize}
    \item Finish writing the thesis plan (phase 1 i.e.\ this document)
    \item Start writing the thesis \textit{`Introduction`} and \textit{`Theoretical Background`} chapters (Start of phase 2)
    \item Create a test LAN environment using a type 2 hypervisor with Virtualbox.
    Also, create a config for such an environment using Vagrant (for ease of reproducing the environment or changing it)
    \item In such a network have at least one node as a server, one as a client that produces malicious/non-malicious traffic, and one node to act as a traffic sniffer
    \item The traffic is captured using already existing tools such as Wireshark, tcpdump and/or other tools
    \item Create a system for analyzing the captured traffic and output a score indicating in relative terms how safe/unsafe the traffic is suspected to be.
    The captured traffic can be analyzed on the host machine or anywhere else
    \item Fine-tune the system and improve it;
    repeat if applicable
    \item If the system is sufficient for the project, start analyzing the results.
    Also, inform the thesis supervisor about phase 2 completion (Execution completed)
    \item Start writing \textit{`Implementation`} and \textit{`Results`} chapters
    \item Start writing `Summary and Conclusions` chapter
    \item Start writing `Abstract`
    \item Inform the thesis supervisor about phase 3 start (Thesis ready for evaluation)
    \item Enroll to maturity test regarding thesis
    \item Submit thesis for language revision after thesis advisor approval
    \end{itemize}

    \section{Task Estimates}\label{sec:tasks}
    The tasks mentioned in \autoref{sec:approach} are divided into three main phases with the following estimated completion dates:
    \renewcommand{\theenumi}{\roman{enumi}}%
    \begin{enumerate}
    \item \textbf{Phase 1}: Planning 2022-11-01 -- 2022-11-04
    \item Collect reference materials 2022-11-16 -- 2022-11-30
    \item \textbf{Phase 2}: Execution 2022-12-01 -- 2023-04-01
    \item Finish remaining chapters 2023-04-02 -- 2023-04-15
    \item \textbf{Phase 3}: Thesis Evaluation 2023-04-16 -- 2023-05-16
    \item Post-evaluation phase 2023-05-17 -- 2023-05-31
    \end{enumerate}
\end{document}
