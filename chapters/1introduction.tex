\chapter{Introduction}\label{ch:introduction}

In the world of Information Technology, cyberattacks are quite ubiquitous.
A cursory look at the news might even suggest that we should expect a service or a device we use to come under a successful attack if it has not already been so.
It therefore becomes quite important to determine with a certain confidence that a cyberattack took place.

The main goal of this thesis is to investigate how a forensic system for detecting malicious network traffic could be implemented in practice.
In this research, the aim is to evaluate such a system's effectiveness and whether it could be useful as a low-cost forensic security solution.
To achieve this goal, the thesis will first briefly analyze the current state of tools available for detecting malicious traffic.
Furthermore, some background concerning malicious network traffic and supervised machine learning methods will be explored and discussed.
The thesis will then finally examine the system implemented, and draw conclusions from the implementation result.

The thesis contains a limited scope—it does not compare the implemented system with existing paid and free solutions for which there are many.
Moreover, it does not take into account usage of the forensic system in an active real-world network.
Therefore, the effectiveness of the system is speculated since the generation of malicious network traffic is in a controlled setting.
