\clearpage%if the chapter heading starts close to bottom of the page, force a line break and remove the vertical vspace
\vspace{21.5pt}


\chapter{Conclusion}\label{ch:conclusions}

The primary goals of this thesis were to detect malicious network traffic,
investigate the practical implementation of a forensic system for this purpose,
and evaluate its effectiveness as a low-cost security solution.
Through the application of supervised machine learning methods,
it was demonstrated that it is possible to detect malicious traffic in a reliable way.

Regarding the practical implementation process, it was heavily reliant on the availability of open datasets.
Without available open datasets, the process would have been arduous.
The training process itself was fairly straight-forward with most time spent on tuning \gls{hyper-param}.
Therefore, the goal of practical implementation is indeed realistic to achieve when quality datasets are obtainable.

As to the effectiveness as a low-cost security solution,
it depends on the resources available to an individual or an organization.
For entities with considerable capital and resources,
it can be feasible to allocate some of those resources to train such network forensic models
and even create datasets that are used for that purpose.
However, in a market economy it can be seen as a waste of resources to do so,
therefore, it is more probable that a few organizations or companies sell their own solutions to others.
Nevertheless, it can be assumed that organizations with high-security profiles, for example,
government agencies and military facilities are creating
and using their own network forensic solutions which are considered as a low-cost security solution to them.

While this thesis has achieved its main goal, it is important to recognize some limitations as well.
One weakness is the limited the number of attack type scenarios that were examined,
both in the actual dataset and the ones generated for the final evaluation;
more attack data would have raised the confidence of the final results.
Another limitation is
that the trained models were not tested in a real-world network which could have brought interesting observations.

In closing,
one area of further research recommended is the development of open datasets
that can be used to create network forensic solutions.
Most of the current available datasets rely
on creating malicious traffic in a virtualized environment and labelling them appropriately;
this can be considered a cumbersome and time-consuming process.
It could be feasible to crowdsource the creation of such datasets with multiple organizations
contributing their traffic captures and logs.
The inherent privacy issues of such an approach could be alleviated with some form of anonymization techniques,
perhaps again with the help of machine learning methods.
