\clearpage%if the chapter heading starts close to bottom of the page, force a line break and remove the vertical vspace
%\vspace{21.5pt}

\chapter{Current State Analysis}\label{ch:current-state-analysis}

The field of IT is very important to the modern economic infrastructure since it helps facilitate nearly all modern
commerce, logistics, business, research, healthcare, and more.
Its crucial role has made it a highly valuable target for cybercriminals hoping to gain money and influence.
As stated previously, having a way to analyze and demonstrate with a degree of certainty that a cyberattack took place is one of the goals in forensic analysis of network traffic.

In the current technology tools offering, there are some tools that can determine whether a cyberattack took place.
These tools are usually classed as \gls{siem}, \gls{ips}, and \gls{ids}.

\gls{siem} tools are usually mostly employed as a `Software as a service' (SaaS) solution in a Cloud platform to
solve security and compliance requirements as required by specific industries.
Examples of such compliance standards include the Sarbanes-Oxley Act (SOX), General Data Protection Regulation (GDPR), and others.
\gls{siem}s are good at determining and alerting the presence of cyberattacks;
however, they are a fairly expensive investment for small organizations or individuals with smaller needs in terms of data traffic volume and functionality.

On the other hand, \gls{ips}es and \gls{ids}es are less expensive in comparison to \gls{siem} solutions.
They are employed as either an on-premise or a Cloud solution in the form of hardware or software to detect cyberattacks,
and in the case of \gls{ips} also prevent detected cyberattacks.
Hardware-based \gls{ips}es and \gls{ids}es are generally faster and more expensive compared to software-based counterparts.

The aforementioned tools have something in common—they require a non-trivial investment in money and resources.
Therefore, this thesis will examine whether building a similar system is practical and achievable in terms of effort and cost.
It will do this by evaluating a software-based forensic system that was built by making use of available open datasets to train machine learning models to identify whether a particular piece of network traffic is malicious or benign.
The implemented system is not a drop-in replacement for the mentioned tools, rather it is a proof of concept of building an `\gls{ids}-like' functionality with only open data and freely available tools.
